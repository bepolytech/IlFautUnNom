\documentclass[10pt]{beamer}
\usetheme{Boadilla}

\usecolortheme{whale}
\usecolortheme{orchid}

\usepackage[utf8]{inputenc}
\usepackage[francais]{babel}
\usepackage[T1]{fontenc}
\usepackage{amsmath}
\usepackage{amsfonts}
\usepackage{amssymb}
\usepackage{graphicx}

%% Test
\newcommand\boxmath[1]{$\rouge{\boxed{#1}}$}
\usepackage{tcolorbox}

%% Couleurs %%
\usepackage{xcolor}
\definecolor{bleu}{RGB}{14, 68, 175}
\definecolor{bleu3}{RGB}{222, 233, 255 }
\definecolor{orange2}{RGB}{255, 216, 154}
\definecolor{rouge}{RGB}{201, 0, 0}
\definecolor{vert}{RGB}{14, 137, 0}
\definecolor{gris}{RGB}{222,230,230}
\newcommand\rouge[1] {{\color{rouge}{#1}}}
\newcommand\bleu[1] {{\color{bleu}{#1}}}
\newcommand\green[1]{{\color{vert}{#1}}}

\graphicspath{{images/}}
\newcommand\clsep {CL_\text{sep} }
\author{Sami AS}
\title{Résumé Analyse II chapitre 16}

\begin{document}

\begin{frame}
\titlepage
\end{frame}

\begin{frame}
\tableofcontents
\end{frame}

\section{CL séparées et Théorème de l'Alternative}
\begin{frame}{CL séparées}
\begin{center}
$\left\{ \begin{array}{l}
(\text{EDL\bleu{H}}) \qquad L(y) := a_0 y'' + a_1 y' + a_2 = \bleu{0} \\
(\text{CL\green{H}}) \qquad \left\{\begin{array}{l}
(CL_a) \quad cl_a(y) := \alpha_1 y(a) + \alpha_2 y'(a) = \green{0} \\
(CL_b) \quad cl_b(y) := \beta_1 y(b) + \beta_2 y'(b) = \green{0} 
\end{array}\right.
\end{array} \right.$
\end{center}
\begin{itemize}
\item Conditions aux limites séparées = paire de combilis.
\item On considère les coefficients des combili \textbf{non-nuls}, donc si les $ \clsep$ sont \textbf{homogènes}, alors on a des choses sucrées.
\begin{itemize}
\item Soient $y_1$ et $y_2$ deux \rouge{fonctions qui satisfont à $(CL_a)$} ($\in \mathit{Cl}_a$) $$ \Rightarrow \left\{ \begin{array}{l} 
cl_a(y_1) = 0 \\
cl_a(y_2) = 0 
\end{array}\right.\qquad \Rightarrow W(y_1, y_2)_{\vert_a} = 0$$
\end{itemize} 
\item[$\Rightarrow$] Grosse propriété donc : le Wronskien de deux fonctions satisfaisant une des CL de $\clsep$ \textbf{est nul}.
\item Plus concrètement, pour une $\clsep$, on a  $$W(y_1, y_2)_{\vert_a} =  W(y_1, y_2)_{\vert_b} = 0 $$
\end{itemize}
\end{frame}

\begin{frame}{Théorème de l'Alternative}
\begin{alertblock}{Théorème de l'Alternative}
\begin{center}
$(\text{P\bleu{H}}) \quad \left\{ \begin{array}{l}
L(y) = \bleu{0} \\
cl_1(y) = 0  \\
 \qquad\vdots \\
cl_n(y) = 0
\end{array}\right. \qquad 
(\text{P\green{nH}}) \quad \left\{ \begin{array}{l}
L(y) = \green{f} \\
cl_1(y) = d_1  \\
 \qquad\vdots \\
cl_n(y) = d_2
\end{array}\right.$
\end{center}

\begin{itemize}
\item Le PnH admet une et une seule solution $\iff$ le PH n'a que la solution triviale $\iff det(M_{\mathit{Cl}}) \neq 0 $.
\item Si le PH admet une autre solution que la solution triviale, alors soit le PnH a une infinité de solutions, soit aucune.
\end{itemize} 
\end{alertblock}
Avec $M_{\mathit{Cl}} = \begin{pmatrix}
\mathit{Cl}_1(y_1) & \cdots & \mathit{Cl}_1 (y_n) \\
\vdots & \ddots & \vdots \\
\mathit{Cl}_n(y_1) & \cdots & \mathit{Cl}_n (y_n) 
\end{pmatrix}$
\end{frame}
\section{Identité de Green et opérateur réduit, autoadjoint}
\begin{frame}{Opérateur réduit et identité de Green}
\begin{block}{Opérateur différentiel réduit}
Un opérateur réduit est un opérateur qui s'écrit sous la forme $$L:= -DpD + q$$
et tout opérateur différentiel $P(D) = PD^2 + QD + R$ peut s'écrire sous une forme réduite avec $$ p = e^{\int \frac{Q}{P}}, \qquad q = -\frac{R}{P} e^{\int \frac{Q}{P}}.$$
\end{block}

\begin{block}{Identité de Green}
L'identité de Green stipule que pour un opérateur \textbf{réduit} $L$, on a
$$\forall \rouge{x} \in [a,b] : \int^{\rouge{x}} _{\color{black}{a}}   (L(u)v - uL(v)) = \left[pW(u,v)\right] ^{\rouge{x}} _{\color{black}{a}}  $$
Utile si $u,v$ satisfont à des $\clsep$ !
\end{block}
\end{frame}
\begin{frame}{Opérateur autoadjoint}
\underline{Définition} : $ F \text{ est autoadjoint dans $CL$} \iff  \forall u,v \in CL : \langle Fu, v \rangle = \langle u, Fv \rangle $
\begin{block}{Opérateur différentiel réduit autoadjoint}
Si l'on restreint $L$ réduit à la classe de fonctions $Cl$, alors l'opérateur $L$ est \textbf{autoadjoint}. C'est dû à la remarque en bas du slide précédent : le Wronskien deux fonctions satisfaisant une Cl est nul, et donc le crochet de l'identité de Green est nul !
$$\boxed{\mathcal{L} \equiv L_{CL_{[a,b]}} \text{autoadj : } \quad \forall u,v : \int_a ^b L(u)v = \int_a ^b uL(v)} $$
\end{block}
\end{frame}
\section{Problème de Sturm-Liouville}
\begin{frame}{Problème de Sturm-Liouville}
\begin{alertblock}{Énoncé d'un problème de Sturm-Liouville}
Le problème de \textbf{Sturm-Liouville} consiste à rechercher les solutions \textbf{non-triviales} $(\lambda ,y)$ de 
$$\left\{ \begin{array}{l}
L(y) = \lambda y \\
cl_a(y) = 0 \\
cl_b(y) = 0
\end{array} \right.$$

\end{alertblock}
\begin{block}{Propriétés d'un problème de Sturm-Liouville}
\begin{enumerate}
\item Les \textbf{valeurs propres} sont réelles, simples et admettent chacune une \textbf{fonction propre} réelle
\item Les fonctions propres associées à des valeurs propres différentes sont orthogonales.
\item Les $\lambda_k$ forment une suite croissante vers $+\infty$
\item $(y_n)_{n\in\mathbb{N}_0}$ est un système orthogonal complet dans $CL$
\end{enumerate}

\end{block}
\end{frame}

\end{document}